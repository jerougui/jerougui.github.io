%%%%%%%%%%%%%%%%%%%%%%%%%%%%%%%%%%%%%%%%%
% Twenty Seconds Resume/CV
% LaTeX Template
% Version 1.0 (30/6/20)
%
% Original author:
% Carmine Spagnuolo (cspagnuolo@unisa.it) and with major modifications by 
% Vel (vel@LaTeXTemplates.com), and Harsh (harsh.gadgil@gmail.com) 
% and Rougui (jamal.rougui@gmail.com) :
% + translate to french language
% + add profile section
% + fix some compilation problem and style
%
% License:
% The MIT License (see included LICENSE file)
%
%%%%%%%%%%%%%%%%%%%%%%%%%%%%%%%%%%%%%%%%%

%----------------------------------------------------------------------------------------
%	PACKAGES AND OTHER DOCUMENT CONFIGURATIONS
%----------------------------------------------------------------------------------------
\documentclass[letterpaper]{twentysecondcv} % a4paper for A4
\newfontfamily{\FAA}{[FontAwesome.otf]}
% Command for printing skill overview bubbles
\newcommand\skills{ 
~
	\smartdiagram[bubble diagram]{
        \textbf{DEV}\\\textbf{Logiciels},
        \textbf{~~Full Stack~~}\\\textbf{~~Java~~},
        \textbf{~~Agile~~},
        \textbf{~~~~~~BDD~~~~~~},
        \textbf{~~~Machine~~~}\\\textbf{~~~Learning~~~},
        \textbf{~~~MVC~~~},
        \textbf{~~~IC~~~},
        \textbf{~~~BigData~~~},	
        \textbf{~~~R\&D~~~}
    }
}

\aboutme{\\
 {\large \textbf{11}} ans d'expérience professionnelle dans le métier d'ingénierie informatique chez divers clients (Thales, INSEE, NEXTER, Bouygues, ...) à l'écoute, bon communiquant et force de proposition.
 {\large \textbf{5}} ans d'expérience en R\&D me procurant une forte capacité de trouver des solutions les plus efficaces et une grande envie pour l'innovation et les nouvelles technologies (NTIC).
%Passionné par l'innovation et la veille technique, toujours attiré par des projet innovant et "from skratch" faisant appel à des outils et Frameworks maintenable et libres.
}

% Programming skill bars
\programming{{C++ $\textbullet$ Python  $\textbullet$ Machine Learning / 3}, {JEE $\textbullet$ WS $\textbullet$ \large Front / 3.5}, {Java $\textbullet$ JavaFx $\textbullet$ RxJava $\textbullet$ Hibernate / 5}}

% Projects text
\education{
\textbf{Doctorat, Sciences d'ingénieries}\\
Uiversité Mohammed V - Université Nantes \\
2004 - 2008 | Nantes, France

\textbf{M2., Info\&Télécom} \\
Université Mohammed V de Rabat \\
2002 - 2004 | Rabat, Maroc
}

%----------------------------------------------------------------------------------------
%	 PERSONAL INFORMATION
%----------------------------------------------------------------------------------------
% If you don't need one or more of the below, just remove the content leaving the command, e.g. \cvnumberphone{}

\cvname{JAMAL-E ROUGUI} % Your name
\cvjobtitle{ Senior Backend  Dev} % Job
% title/career

\cvlinkedin{/in/jamal-rougui-45462624/}
\cvgithub{jerougui/jerougui.github.io}
\cvnumberphone{(+33) 6 60 444 936} % Phone number
\cvsite{} % Personal website
\cvmail{jamal.rougui@gmail.com} % Email address

%----------------------------------------------------------------------------------------

\begin{document}

\makeprofile % Print the sidebar

%----------------------------------------------------------------------------------------
%	 EXPERIENCE
%----------------------------------------------------------------------------------------

\section{Expériences}

\begin{twenty} % Environment for a list with descriptions
\twentyitem
    	{Jun 2015 -}
		{à ce jour}
        {Lead Tech \& Agile master}
        {\href{https://www.thalesgroup.com/fr}{Thales SIX GTS France}}
        {}
        {\begin{itemize} 		
      	\item Développement client lourd, autour des outils de tests et de validation en mode Feature Team.
        \item Sprint planning, revu, estimation, découpage, faisabilité
        \item développement et test, qualité, intégration continue
        \item force de proposition et amélioration continue
        \item Démo, review, rétrospective, repport
        \end{itemize}}
        \\
	\twentyitem
    	{Oct 2014 -}
		{Jun 2015}
        {Lead Tech \& Proxy Chef de Projet}
        {\href{https://www.insee.fr/fr/accueil}{INSEE, SII-Nantes}}
        {}
        {
        {\begin{itemize}
		\item Etudes de faisabilité et sépecification formelle
		\item Migration IE to Firefox, et mise en place du SSO (JEE)
		\item Maintenance Adaptative et évolutive d'application (ASP)
		\item Maintenance Adaptative du Socle Java et Atelier de développement
    	\end{itemize}}
        }
     \\
     \twentyitem
   		{Avr 2013 -}
		{Jun 2015}
        {Développeur confirmé Java/Swing}
        {\href{}{HMX-HYPERVISION}}
        {}
        {
        \begin{itemize}
        \item Maintenance évolutive et corrective des logiciels vote électronique
        \item Développement des solutions
        \item Développement tests et packaging multi OS et support
    	\end{itemize}
    	}
	\\   
		\twentyitem
		{Mai 2014 -}
		{Sep 2014}
		{Développeur JEE}
		{\href{https://www.nexter-group.fr/}{NEXTER}}
		{}
		{
		\begin{itemize}
		\item Migration d'un prototype client lourd (java Swing) en client leger
		\item Conception d'architecture logicielle dans un environnement JEE
		\item Amélioration continue et metrique de la qualité du code
		\item Développement, test, optimisation des performances BDD.
		\end{itemize}
		}       
       \\
	    \twentyitem
	    {Oct 2011 -}
	    {Nov 2012}
	    {Développeur Java/C++}
	    {\href{https://www.jobs.bouyguestelecom.fr/}{CDN - Bouygues Telecom}}
	    {}
	    {
	    \begin{itemize}
	    \item Maintenance adaptative et évolutive intervention dans toutes les étapes du cycle en V (specification, développement, tests, support et maintenance) du SI Facturation et comptabilité auxiliaire
	    \item Rédaction de spécifications techniques
	    \item Développement Java/C++/PLSQL/KSH
	    \item Packaging et livraison	
	    \end{itemize}
	    }
    
	   \\
	    \twentyitem
	    {Mai 2011 -}
	    {Oct 2012}
	    {Développeur Java}
	    {\href{}{E-Map, Ecole Polytech'Nantes}}
	    {}
	    {
	    \begin{itemize}
	    \item Développement d'un moteur de recherche sémantique des données (profil a caractère professionnel)
	    \item Implémentation des solutions de traitement des grands volumes de données, classification et recommandation	
	    \end{itemize}
	    }

	   \\
	    \twentyitem
	    {Jan 2011 -}
	    {Avr 2012}
	    {Développeur C++}
	    {\href{}{RnD-Technologies}}
	    {}
	    {
	    \begin{itemize}
		\item Conception et réalisation d'une application de gestion et de service de restauration d'entreprise.
		\item Proposition et mise en oeuvre des solutions logicielles	
	    \end{itemize}
	    }

	   \\
		\twentyitem
		{Nov 2009 -}
		{Aoû 2010}
		{Attaché Temporaire Enseignement et Recherche (ATER)}
		{\href{http://www.lina.univ-nantes.fr/}{LINA}}
		{}
		{
		\begin{itemize}
		\item Enseignement de l'informatique au niveau L1/L2
		\item Algorithmique et programmation
		\item Ingénierie du Web et Bases de données relationnelle
		\end{itemize}
		}
	%\twentyitem{<dates>}{<title>}{<location>}{<description>}
\end{twenty}

%----------------------------------------------------------------------------------------
%	 RESEARCH
%----------------------------------------------------------------------------------------
\section{Recherche}
\begin{twenty}
	\twentyitem
    	{2004 - 2009}
		{}
        {R\&D. en Informatique}
        {\href{https://www.univ-nantes.fr/}{ESIGETEL \& Université de Nantes}}
        {}
	       {
	        \textbf{PostDoc}: Reconnaissance des sons de vie courante en environnement intelligent pour la sante - Robot CompanionAble. \\
	       	\textbf{Thèse}: Indexation de documents audio: cas des gros volumes de données.
	        {
	        %\begin{itemize}
	        %\item 
			%\end{itemize}
			}
	       }
  
\end{twenty}

%section{Publications}
% Jamal Rougui, Marc Gelgon, Mohamed Rziza, José Martinez, Driss Aboutajdine, “Fast incremental clustering of Gaussian mixture speaker models for scaling up retrieval in on-line broadcast ” Speech and Signal Processing (ICASSP 2006), 2006, Toulouse, France. pp.521-524. \vspace{2mm}

\end{document} 
