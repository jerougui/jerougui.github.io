%%%%%%%%%%%%%%%%%%%%%%%%%%%%%%%%%%%%%%%%%
% Twenty Seconds Resume/CV
% LaTeX Template
% Version 1.0 (14/7/16)
%
% Original author:
% Carmine Spagnuolo (cspagnuolo@unisa.it) with major modifications by 
% Vel (vel@LaTeXTemplates.com) and Harsh (harsh.gadgil@gmail.com)
%
% License:
% The MIT License (see included LICENSE file)
%
%%%%%%%%%%%%%%%%%%%%%%%%%%%%%%%%%%%%%%%%%

%----------------------------------------------------------------------------------------
%	PACKAGES AND OTHER DOCUMENT CONFIGURATIONS
%----------------------------------------------------------------------------------------
\documentclass[letterpaper]{twentysecondcv} % a4paper for A4
\newfontfamily{\FAA}{[FontAwesome.otf]}
% Command for printing skill overview bubbles
\newcommand\skills{ 
~
	\smartdiagram[bubble diagram]{
        \textbf{DEV}\\\textbf{Logiciels},
        \textbf{Full Stack}\\\textbf{Java},
        \textbf{~~~~~~~~Agile~~~~~~~~~},
        \textbf{~~~~~~Bases~~~~~~}\\\textbf{~~Données~~},
        \textbf{~~~~~R\&D~~~~~}
    }
}

% Programming skill bars
\programming{{C $\textbullet$ C++  $\textbullet$ Matlab / 3}, {JavaScript $\textbullet$ SQL $\textbullet$ \large Python / 3.5}, {JEE $\textbullet$ Java8 $\textbullet$ JavaFx / 5}}

% Projects text
\education{
\textbf{Doctorat, Sciences d'ingénieries}\\
Spécialisation: Indexation multimédia \\
Uiversité Mohammed V - Université Nantes \\
2004 - 2008 | Nantes, France

\textbf{M2., Info\&Télécom} \\
Université Mohammed V de Rabat \\
2002 - 2004 | Rabat, Maroc
}

%----------------------------------------------------------------------------------------
%	 PERSONAL INFORMATION
%----------------------------------------------------------------------------------------
% If you don't need one or more of the below, just remove the content leaving the command, e.g. \cvnumberphone{}

\cvname{JAMAL-E ROUGUI} % Your name
\cvjobtitle{ Lead Tech Java } % Job
% title/career

\cvlinkedin{/in/jamal-rougui-45462624/}
\cvgithub{jerougui/jerougui.github.io}
\cvnumberphone{(+33) 6 60 444 936} % Phone number
\cvsite{} % Personal website
\cvmail{jamal.rougui@gmail.com} % Email address

%----------------------------------------------------------------------------------------

\begin{document}

\makeprofile % Print the sidebar

%----------------------------------------------------------------------------------------
%	 EXPERIENCE
%----------------------------------------------------------------------------------------

\section{Expériences}

\begin{twenty} % Environment for a list with descriptions
\twentyitem
    	{Juin 2015 -}
		{à ce jour}
        {Developpeur confirmé Java8, JavaFX, JavaRx}
        {\href{https://www.thalesgroup.com/fr}{Thales SIX GTS France}}
        {}
        {\begin{itemize} 		
      	\item Développement client lourd, autour des outils de tests et de validation en mode Feature Team.
        \item Sprint planning, revu, estimation, découpage, faisabilité
        \item développement et test, qualité, intégration continue
        \item Démo, review, rétrospective, repport
        \end{itemize}}
        \\
	\twentyitem
    	{Oct 2013 -}
		{Juin 2015}
        {Proxy Chef de Projet \& Référent technique}
        {\href{https://www.insee.fr/fr/accueil}{INSEE, SII-Nantes}}
        {}
        {
        {\begin{itemize}
		\item Etudes de faisabilité et sépcification 
		\item Migration IE to Firefox, et mise en place du SSO (JEE)
		\item Maintenance Adaptative et evolutif d'application (ASP)
		\item Maintenance Adaptative du Socle Java et Atelier de développement (AdminStudio, InstallShiled, NSIS)
    \end{itemize}}
        }
    \\   
    \twentyitem
   		{Mai 2014 -}
		{Sep 2014}
        {Lead technique JEE}
        {\href{https://www.nexter-group.fr/}{NEXTER}}
        {}
        {
        {\begin{itemize}
     		\item Migration d'un prototype client lourd (java Swing) en client leger (Application Web) avec prise en compte
     		des complements securitaires et d'integration au SI existant (Active Directory).
     		\item Concepteur d'architecture logicielle dans un environnement JEE
     		\item Developpement, test, Optimisation des performances BDD.
    \end{itemize}}
        }
     \\
     \twentyitem
   		{Avril 2013 -}
		{Juin 2015}
        {Referent technique (JAVA/SWING)}
        {\href{}{HMX-HYPERVISION}}
        {}
        {
        \begin{itemize}
        	\item Maintenance evolutive et corrective des logiciels de vote electronique Hypermaster
        	\item Etudes de faisabilite et traduire les besoins client en evoluions avec chiffrage
        	\item Développement tests et packaging, support
    \end{itemize}
    	}
        
       \\
        \twentyitem
        {Oct 2011 -}
        {Nov 2012}
        {Développeur Java/C++}
        {\href{https://www.jobs.bouyguestelecom.fr/}{CDN - Bouygues Telecom}}
        {}
        {
        	\begin{itemize}
        		\item Maintenance adaptative et evolutive intervention dans toutes les etapes du cycle en V (specification, développement, tests, support et maintenance) du ST Facturation et comptabilite auxiliaire
        		\item Rédaction de spécifications techniques
        		\item Développement Java/JEE/PLSQL/KSH
        		\item Packaging et livraison	
        	\end{itemize}
        }
    
           \\
    \twentyitem
    {Jan 2011 -}
    {Avr 2012}
    {Développeur C++}
    {\href{}{RnD-Technologies}}
    {}
    {
    	\begin{itemize}
		\item Conception et realisation d'un systeme d'information (Si) dans le domaine de service de restauration
		d'entreprise.
		\item Rédaction de specifications techniques
		\item Développement
		\item Integration solution logicielle	
    	\end{itemize}
    }

       \\
	\twentyitem
	{Nov 2009 -}
	{Aout 2010}
	{Attache Temporaire Enseignement et Recherche (ATER)}
	{\href{http://www.lina.univ-nantes.fr/}{LINA}}
	{}
	{
		\begin{itemize}
			\item Enseignement/Encadrement en informatique niveau Licence 1 et 2
			\item Algorithmique et programmation
			\item Ingenierie du Web et Bases de donnees relationelle
		\end{itemize}
	}
	%\twentyitem{<dates>}{<title>}{<location>}{<description>}
\end{twenty}

%----------------------------------------------------------------------------------------
%	 RESEARCH
%----------------------------------------------------------------------------------------
\section{Recherche}
\begin{twenty}
	\twentyitem
    	{2004 - 2009}
		{}
        {R\&D. en Informatique}
        {\href{https://www.univ-nantes.fr/}{Université de Nantes, ESIGETEL}}
        {}
        {
        \textbf{PostDoc}: Reconnaissance des sons de vie courante en environnement intelligent pour la sante - Robot CompanionAble. \\
       	\textbf{Thèse}: Indexation de documents audio: cas de grands volume de données.
        {
        %\begin{itemize}
        %\item Proposed a stepwise deterministic method to integrate datasets without labeled data. The method performs comparably with a method that incorporates a Support Vector Machine
        %\item Prepared a longitudinal dataset to enable comprehensive analyses about WWI Canadian society and military, seeding further research
        %\item \textbf{Tools}: R, Python, scikit-learn, pandas \vspace{2mm}
		%\end{itemize}
	}
        }
  
\end{twenty}

%section{Publications}
%L. Antonie, H. Gadgil, G. Grewal, and K. Inwood, “Historical Data Integration - A Study %of WWI Canadian Soldiers,” in 2016 IEEE 16th International Conference on Data Mining %Workshops (ICDMW), pp. 186-193, IEEE, 2016. \vspace{2mm}

\end{document} 
